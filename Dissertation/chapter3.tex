\chapter{Третья глава}
\label{ch:ch3}

\section{Guide for the amsmath Package}
The \textbf{amsmath} package provides a number of additional displayed equation structures beyond the ones provided in basic \LaTeX. The augmented set includes:
\begin{verbatim}
	equation 	equation* 	align 		align*
	gather 		gather* 	alignat 	alignat*
	multline 	multline* 	flalign 	flalign*
	split
\end{verbatim}

better to use \verb*|align| or \verb*|equation+split|

The \textbf{split} environment is a special subordinate form that is used \textcolor{red}{only inside one of the others, and the split environment provides no numbering.} \textcolor{magenta}{It cannot be used inside multline},  usually an \verb*|equation,align,gather| environment, however \textbf{split} supports only one alignment (\&) column; if more are needed, \textbf{aligned} or \textbf{alignedat} should be used. 

\subsection{Single equations}
The \textbf{equation} environment is for a single equation with an automatically generated number. 
The wrapper \verb*|\[...\]| is equivalent to \textbf{equation*}.

\subsection{Split equations without alignment}
%Môi trường đa dòng là một biến thể của môi trường phương trình được sử dụng cho các phương trình không vừa trên một dòng.
The \textbf{multline} environment is a variation of the equation environment used for equations that \textcolor{red}{don’t fit on a single line}. \\
Example for equation + split
\begin{equation}\label{xx}
	\begin{split}
		a& =b+c-d\\
		& \quad +e-f\\
		& =g+h\\
		& =i
	\end{split}
\end{equation}
Example for gather 
\begin{gather}
	a_1=b_1+c_1\\
	a_2=b_2+c_2-d_2+e_2
\end{gather}
Example for align
\begin{align}
	a_1& =b_1+c_1\\
	a_2& =b_2+c_2-d_2+e_2
\end{align}
\begin{align}
	a_{11}& =b_{11}& a_{12}& =b_{12}\\
	a_{21}& =b_{21}& a_{22}& =b_{22}+c_{22}
\end{align}

\subsection{Split equations with alignment}
%Giống như nhiều dòng, môi trường phân tách dành cho các phương trình đơn quá dài để vừa trên một dòng và do đó phải được chia thành nhiều dòng.
Like multline, the split environment is for single equations that are \textcolor{blue}{too long to fit on one line} and \textcolor{red}{hence must be split into multiple lines.}
\begin{equation}\label{e:barwq}
	\begin{split}
		H_c&=\frac{1}{2n} \sum^n_{l=0}(-1)^{l}(n-{l})^{p-2}
		\sum_{l _1+\dots+ l _p=l}\prod^p_{i=1} \binom{n_i}{l _i}\\
		&\quad\cdot[(n-l )-(n_i-l _i)]^{n_i-l _i}\cdot
		\Bigl[(n-l )^2-\sum^p_{j=1}(n_i-l _i)^2\Bigr].
	\end{split}
\end{equation}

\subsection{Equation groups with mutual alignment}
\begin{align}
	x 	&=y 		& X		&=Y 	& a		&=b+c\\
	x’	&=y’ 		& X’	&=Y’ 	& a’	&=b\\
	x+x’&=y+y’ 		& X+X’	&=Y+Y’ 	& a’b	&=c’b
\end{align}
Line-by-line annotations on an equation can be done by judicious application of
\text inside an align environment:
\begin{align}
	x& = y_1-y_2+y_3-y_5+y_8-\dots 	&& \text{by \eqref{e:barwq}}\\
	& = y’\circ y^* 				&& \text{by \eqref{xx}}\\
	& = y(0) y’ 					&& \text{by Axiom 1.}
\end{align}

\subsection{Alignment building blocks}
\textbf{gathered}, \textbf{aligned}, and \textbf{alignedat} 
\begin{equation*}
	\left.\begin{aligned}
		B’&=-\partial\times E,\\
		E’&=\partial\times B - 4\pi j,
	\end{aligned}
	\right\}
	\qquad \text{Maxwell’s equations}
\end{equation*}

\begin{equation}
	P_{r-j}=
	\begin{cases}
		0					& \text{if $r-j$ is odd},\\
		r!\,(-1)^{(r-j)/2}	& \text{if $r-j$ is even}.
	\end{cases}
\end{equation}

\subsection{Matrices}
The pmatrix, \textcolor{red}{bmatrix}, Bmatrix, vmatrix and Vmatrix have (respectively) \verb*|( ), [ ], { }|, | | and $\| \|$ delimiters built in.

To produce a small matrix suitable for use in text, there is a smallmatrix
environment e.g. 
$\bigl( \begin{smallmatrix}
	a&b\\ c&d
\end{smallmatrix} \bigr)$

\begin{equation}
	\begin{pmatrix} D_1t&-a_{12}t_2&\dots&-a_{1n}t_n\\
		-a_{21}t_1&D_2t&\dots&-a_{2n}t_n\\
		\hdotsfor[2]{4}\\ %tạo ra dấu chấm ở 4 cột, khoản cách ... là 2.
		-a_{n1}t_1&-a_{n2}t_2&\dots&D_nt
	\end{pmatrix}
\end{equation}

\subsection{Dots}
\verb*|\ldots \dots \cdots \dotsc \dotsb \dotsm \dotsi \dotso|
\begin{align*}
	\ldots \; \dots \; \cdots \; \dotsc \; \dotsb \; \dotsm \; \dotsi \; \dotso
\end{align*}
\verb*|\cdots \ddots \ldots \dots \vdots|
\begin{align}
	\cdots \quad 
	\ddots \quad \ldots \quad 
	\dots \quad \vdots 
\end{align}
\subsection{Roots}
$\sqrt[\beta]{k}$ vs $\sqrt[\leftroot{-2}\uproot{2}\beta]{k}$

\subsection{Boxed formulas}
\begin{equation}
	\boxed{\eta \leq C(\delta(\eta) +\Lambda_M(0,\delta))}
\end{equation}

\subsection{Over and under arrows}
Basic \LaTeX:
\begin{verbatim}
	\overleftarrow 		\underleftarrow
	\overrightarrow 	\underrightarrow
	\overleftrightarrow \underleftrightarrow
\end{verbatim}
\begin{align}
	\overleftarrow{a+b} 		\quad \underleftarrow{a+b} \\
	\overrightarrow{a+b} 		\quad \underrightarrow{a+b} \\
	\overleftrightarrow{a+b} 	\quad \underleftrightarrow{a+b}
\end{align}
Extensible arrows (amsmath package): \verb*|\xleftarrow and \xrightarrow|
\[\xleftarrow{n+\mu-1}\quad \xrightarrow[T]{n\pm i-1}\]

\subsection{Over- and Underlining}
\begin{align*}
	\underline{text} \quad
	\overline{text} \quad
	\underbrace{math} \quad
	\overbrace{math}
\end{align*}
\begin{equation}
	1+1/2+\underbrace{1/3+1/4}_{>1/2}+
	\underbrace{1/5+1/6+1/7+1/8}_{>1/2}+\cdots
\end{equation}

\subsection{Fractions and related constructions}
\begin{equation}
	\frac{1}{k}\log_2 c(f)\;\tfrac{1}{k}\log_2 c(f)\;
	\sqrt{\frac{1}{k}\log_2 c(f)}\;\sqrt{\dfrac{1}{k}\log_2 c(f)}
\end{equation}
\begin{equation}
	2^k-\binom{k}{1}2^{k-1}+\binom{k}{2}2^{k-2}
\end{equation}

\subsection{Delimiters}
\begin{align}
	\left[\sum_i a_i \left\lvert\sum_j x_{ij}\right\rvert^p\right]^{1/p} \qquad \text{vs.} \qquad
	\biggl[\sum_i a_i\Bigl\lvert\sum_j x_{ij}\Bigr\rvert^p\biggr]^{1/p}
\end{align}
\begin{align*}
	\left((a_1 b_1) - (a_2 b_2)\right)
	\left((a_2 b_1) + (a_1 b_2)\right)
	\quad\text{versus}\quad
	\bigl((a_1 b_1) - (a_2 b_2)\bigr)
	\bigl((a_2 b_1) + (a_1 b_2)\bigr)
\end{align*}

\textbf{Vertical bar notations}: \verb*|\lvert, \rvert, \lVert, \rVert|
\begin{align}
	\lvert -123 \rvert \qquad 	\abs{-123} \\
	\lVert AAA \rVert  \qquad	\norm{AAA}
\end{align}

\subsection{The text command}

\begin{align}
	f_{[x_{i-1},x_i]} \text{ is monotonic,}	\quad i = 1,\dots,c+1 \\
	\partial_s f(x) = \frac{\partial}{\partial x_0} f(x)\quad
	\text{for $x= x_0 + I x_1$.}
\end{align}

\subsection{Integrals and sums}
\begin{align}
	\sum_{\substack{
			0\le i\le m\\
			0<j<n}}
	P(i,j) \qquad
	\sum_{\begin{subarray}{l}
			i\in\Lambda\\ 0<j<n
	\end{subarray}}
	P(i,j)
\end{align}