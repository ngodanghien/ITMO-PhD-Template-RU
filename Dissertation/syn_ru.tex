\counterwithout{figure}{chapter}
\counterwithout{table}{chapter}
\renewcommand{\figurename}{Рисунок}
\renewcommand{\tablename}{Таблица}
\renewcommand\thesubfigure{\asbuk{subfigure}}

\chapter*{Реферат}
\addcontentsline{toc}{chapter}{Реферат} 

\begin{center}
    Общая характеристика диссертации
\end{center}

\begin{figure}
    \centering
    \includegraphics[width=0.6\linewidth]{images/knuth}
    \caption{Knuth}
    \label{fig:my_label2}
\end{figure}


\paragraph*{Актуальность.}

\paragraph*{Цель исследования.}
\paragraph*{Научные задачи.}

\paragraph*{Методы исследования.}


\paragraph*{Основные положения, выносимые на защиту.}

\paragraph*{Научная новизна.}

\paragraph*{Теоретическая значимость.}
\paragraph*{Практическая значимость.}
\paragraph*{Достоверность.}
\paragraph*{Аппробация работы.}
Основные результаты диссертации докладывались на следующих конференциях:
\printConferenceRU

\paragraph*{Личный вклад автора.}


\paragraph*{Объём и структура работы.}
Диссертация состоит из~введения,
\formbytotal{totalchapter}{глав}{ы}{}{},
заключения и
\formbytotal{totalappendix}{приложен}{ия}{ий}{}.
%% на случай ошибок оставляю исходный кусок на месте, закомментированным
%Полный объём диссертации составляет  \ref*{TotPages}~страницу
%с~\totalfigures{}~рисунками и~\totaltables{}~таблицами. Список литературы
%содержит \total{citenum}~наименований.
%
Полный объём диссертации составляет
\formbytotal{TotPages}{страниц}{у}{ы}{}, включая
\formbytotal{totalcount@figure}{рисун}{ок}{ка}{ков} и
\formbytotal{totalcount@table}{таблиц}{у}{ы}{}.
Список литературы содержит
\formbytotal{citenum}{наименован}{ие}{ия}{ий}.




\newpage
\section*{Основное содержание работы}

В Главе~\ref{ch:ch1}...

\section*{Публикации автора по теме диссертации}


Основные результаты по теме диссертации изложены в \theAllMyPapers~публикациях. 
Из них \theScopusPapers~опубликовано в изданиях, индексируемых в базе цитирования Scopus, 
из них \theVakPapers~изданы в журналах, рекомендованных ВАК. 
%%Также имеется 1 свидетельство о государственной регистрации программ для ЭВМ.
%

В международных изданиях, индексируемых в базе данных Scopus, Web of Science:
\printPapperScopus

В изданиях из перечня ВАК РФ:
%\printPapperVak

В иных изданиях:
\printPapperOther

