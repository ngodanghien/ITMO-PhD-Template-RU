\counterwithout{figure}{chapter}
\counterwithout{table}{chapter}
\chapter{Оформление различных элементов}
\label{ch:ch1}

\begin{figure}
    \centering
    \includegraphics[width=0.6\linewidth]{images/knuth}
    \caption{Knuth}
    \label{fig:my_label4}
\end{figure}


\section{Ссылки}

\cite{KKK, Gerasimov2023a, Gerasimov2023b, NikGerbook2022}

\section{Форматирование чисел и размерностей величин}\label{sec:units}

Числа форматируются при помощи команды \verb|\num|:
\num{5,3};
\num{2,3e8};
\num{12345,67890};
\num{2,6 d4};
\num{1+-2i};
\num{.3e45};
\num[exponent-base=2]{5 e64};
\num[exponent-base=2,exponent-to-prefix]{5 e64};
\num{1.654 x 2.34 x 3.430}
\num{1 2 x 3 / 4}.


Обратите внимание, что ГОСТ запрещает использование знака <<->> для обозначения отрицательных чисел
за исключением формул, таблиц и~рисунков.
Вместо него следует использовать слово <<минус>>.

Размерности можно записывать при помощи команд \verb|\si| и \verb|\SI|:
\si{\farad\squared\lumen\candela};
\si{\joule\per\mole\per\kelvin};
\si[per-mode = symbol-or-fraction]{\joule\per\mole\per\kelvin};
\si{\metre\per\second\squared};
\SI{0.10(5)}{\neper};
\SI{1.2-3i e5}{\joule\per\mole\per\kelvin};
\SIlist{1;2;3;4}{\tesla};
\SIrange{50}{100}{\volt}.

\begin{table}
    \centering
    \captionsetup{justification=centering} % выравнивание подписи по-центру
    \caption{Основные величины СИ}\label{tab:unit:base}
    \begin{tabular}{llc}
        \toprule
        Название  & Команда                 & Символ         \\
        \midrule
        Ампер     & \verb|\ampere| & \si{\ampere}   \\
        Кандела   & \verb|\candela| & \si{\candela}  \\
        Кельвин   & \verb|\kelvin| & \si{\kelvin}   \\
        Килограмм & \verb|\kilogram| & \si{\kilogram} \\
        Метр      & \verb|\metre| & \si{\metre}    \\
        Моль      & \verb|\mole| & \si{\mole}     \\
        Секунда   & \verb|\second| & \si{\second}   \\
        \bottomrule
    \end{tabular}
\end{table}

