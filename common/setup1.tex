%%%%%%%%%%%%%%%%%%%%%%%%%%%%%%%%%%%%%%%%%%%%%%%%%%%%%%%%%%%%%%%%%%%%%%%%%%%%%%%%
%%%% Файл упрощённых настроек шаблона, общих для диссертации и автореферата %%%%
%%%%%%%%%%%%%%%%%%%%%%%%%%%%%%%%%%%%%%%%%%%%%%%%%%%%%%%%%%%%%%%%%%%%%%%%%%%%%%%%

%%% Режим черновика %%%
\makeatletter
\@ifundefined{c@draft}{
  \newcounter{draft}
  \setcounter{draft}{0}  % 0 --- чистовик (максимальное соблюдение ГОСТ)
                         % 1 --- черновик (отклонения от ГОСТ, но быстрая
                         %       сборка итоговых PDF)
}{}
\makeatother

%%% Пометки в тексте %%%
\makeatletter
\@ifundefined{c@showmarkup}{
  \newcounter{showmarkup}
  \setcounter{showmarkup}{0}  % 0 --- скрыть пометки
                              % 1 --- показывать пометки
}{}
\makeatother

%%% Использование в pdflatex шрифтов не по-умолчанию %%%
\makeatletter
\@ifundefined{c@usealtfont}{
  \newcounter{usealtfont}
  \setcounter{usealtfont}{1}    % 0 --- шрифты на базе Computer Modern
                                % 1 --- использовать пакет pscyr, при его
                                %       наличии
                                % 2 --- использовать пакет XCharter, при наличии
                                %       подходящей версии
}{}
\makeatother

%%% Использование в xelatex и lualatex семейств шрифтов %%%
\makeatletter
\@ifundefined{c@fontfamily}{
  \newcounter{fontfamily}
  \setcounter{fontfamily}{1}  % 0 --- CMU семейство. Используется как fallback;
                              % 1 --- Шрифты от MS (Times New Roman и компания)
                              % 2 --- Семейство Liberation
}{}
\makeatother


%%% Вывод типов ссылок в библиографии %%%
\makeatletter
\@ifundefined{c@mediadisplay}{
  \newcounter{mediadisplay}
  \setcounter{mediadisplay}{1}   % 0 --- не делать ничего; надписи [Текст] и
                                 %       [Эл. ресурс] будут выводиться только в ссылках с
                                 %       заполненным полем `media`;
                                 % 1 --- автоматически добавлять надпись [Текст] к ссылкам с
                                 %       незаполненным полем `media`; таким образом, у всех
                                 %       источников будет указан тип, что соответствует
                                 %       требованиям ГОСТ
                                 % 2 --- автоматически удалять надписи [Текст], [Эл. Ресурс] и др.;
                                 %       не соответствует ГОСТ
                                 % 3 --- автоматически удалять надпись [Текст];
                                 %       не соответствует ГОСТ
                                 % 4 --- автоматически удалять надпись [Эл. Ресурс];
                                 %       не соответствует ГОСТ
}{}
\makeatother

%%% Предкомпиляция tikz рисунков для ускорения работы %%%
\makeatletter
\@ifundefined{c@imgprecompile}{
  \newcounter{imgprecompile}
  \setcounter{imgprecompile}{0}   % 0 --- без предкомпиляции;
                                  % 1 --- пользоваться предварительно
                                  %       скомпилированными pdf вместо генерации
                                  %       заново из tikz
}{}
\makeatother
