% author: hiennd, March 03 2024
% for english version.
%\newtheorem{theorem}{Theorem}
%\newtheorem{assumption}{Assumption}
%\newtheorem{lemma}{Lemma}
%\newtheorem{proof}{Proof}
%\newtheorem{proposal}{Proposal}
%\newtheorem{proposition}{Proposition}
%\newtheorem{corollary}{Corollary}
%\newtheorem{statement}{Statement}
%\newtheorem{remark}{Remark}
%\newtheorem{definition}{Definition}
%\newtheorem{example}{Example}
%\newtheorem{problem}{Problem}
%\newtheorem{algorithm}{Algorithm}

% for russian, follow as VSPUart stye (VSPU-2024)
%Synopse: \newtheorem{name}[numbered_like]{title}
% default = plain : italic text, extra space above and below
\theoremstyle{definition}% definition : upright text, extra space above and below;

%\newtheorem{theorem-syn-ru}{Теорема}
\newtheorem{assumption-syn-ru}{Допущение} % Предположение or Допущение 1
\newtheorem{assumption-syn-en}{Assumption}
\newtheorem{lemma-syn-ru}{Лемма}
\newtheorem{lemma-syn-en}{Lemma}
\newtheorem{proposition-syn-ru}{Предложение} % maybe
\newtheorem{proposition-syn-en}{Proposition}
\newtheorem{corollary-syn-ru}{Следствие}
\newtheorem{corollary-syn-en}{Corollary}
\newtheorem{statement-syn-ru}{Утверждение}
\newtheorem{statement-syn-en}{Statement}
\newtheorem{remark-syn-ru}{Замечание}
\newtheorem{remark-syn-en}{Remark}
\newtheorem{definition-syn-ru}{Определение}
\newtheorem{definition-syn-en}{Definition}
\newtheorem{example-syn-ru}{Пример}
\newtheorem{example-syn-en}{Example}
%\newtheorem{problem}{Задача}

\newtheorem{theorem}{Теорема}[chapter]
\newtheorem{assumption}{Допущение}[chapter] % Предположение or Допущение 1.1
\newtheorem{lemma}{Лемма}[chapter]
\newtheorem{proposal}{Предложение}[chapter]
\newtheorem{proposition}{Предложение}[chapter] % maybe
\newtheorem{corollary}{Следствие}[chapter]
\newtheorem{statement}{Утверждение}[chapter]
\newtheorem{remark}{Замечание}[chapter]
\newtheorem{definition}{Определение}[chapter]
\newtheorem{example}{Пример}[chapter]
\newtheorem{problem}{Задача}[chapter]
%\newtheorem{algorithm}{Алгоритм}[chapter]
%\newtheorem{iteration}{Итерационная схема}

%\newtheorem{proof}{Доказательство}
%\renewenvironment{proof}{Доказательство}{}
%\renewenvironment{proof}
%{\vspace\topsep\par{Доказательство.\,\ }\ignorespaces}%
%{\noindent\vspace\partopsep}

%\renewenvironment{proof}
%{\vspace\topsep\par{
%		\textbf{Доказательство.}\,\ }\ignorespaces}%
%{\noindent\vspace\partopsep}


%\makeatletter
%\renewenvironment{proof}[1][\proofname]{\par
%	\pushQED{\qed}%
%	\normalfont \topsep6\p@\@plus6\p@\relax
%	\trivlist
%	\item\relax
%	{\itshape
%		#1\@addpunct{.}}\hspace\labelsep\ignorespaces
%}{%
%	\popQED\endtrivlist\@endpefalse
%}
%\makeatother

%\crefformat{chapter}{#2#1#3} % одиночная ссылка с приставкой
%\labelcrefformat{chapter}{#2#1#3} % одиночная ссылка без приставки
%\crefformat{typei}{format}
%\crefformat{chapter}{Chapter~#2#1#3}

\newcommand\norm[1]{\left\lVert#1\right\rVert}
\newcommand\normx[1]{\left\Vert#1\right\Vert}

\def\R{\mathbb{R}}
\def\C{\mathbb{C}}
\def\N{\mathbb{N}}
\def\Z{\mathbb{Z}}
\def\Q{\mathbb{Q}}

\def\L{\mathcal{L}}	% norm L_x
\def\cal#1{\mathcal{#1}} % class menoir don't support \cal.
%\renewcommand\qedsymbol{QED}
%\renewcommand\qedsymbol{$\blacksquare$}
%\def\qed{\hfill$\blacksquare$}
%\def\boxw{\hfill$\Box$}
