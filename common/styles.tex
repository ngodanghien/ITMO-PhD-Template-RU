%%====(NOTE)========================================================================================
%	1. Dãn cách dòng với dòng (giống Line Spacing trong WORD): \DoubleSpacing; \OnehalfSpacing; \setSpacing{}
% 	2. Thụt TAB ở Line đầu tiên sau (Chapter/Section/Subsection): \setlength{\parindent}{2.5em}
%%======(begin)[ĐỊNH DẠNG KHỔ GIẤY]=========================================>>>>>>>>>>>>>>>>>>>
%\usepackage{geometry}
\geometry{a4paper, top=2cm, bottom=2cm, left=2.5cm, right=1cm, nofoot, nomarginpar}
\setlength{\topskip}{0pt}   %размер дополнительного верхнего поля
\setlength{\footskip}{12.3pt} % để xóa warning, согласно https://tex.stackexchange.com/a/334346
%%% Интервалы %%%
%% По ГОСТ Р 7.0.11-2011, пункту 5.3.6 требуется полуторный интервал
%% Реализация средствами класса (на основе setspace) ближе к типографской классике.
%% И правит сразу и в таблицах (если со звёздочкой)
%\DoubleSpacing*     % Двойной интервал
%\OnehalfSpacing*    % Полуторный интервал
\setSpacing{1.5}   % Полуторный интервал, подобный Ворду (возможно, стоит включать вместе с предыдущей строкой)
%%% Выравнивание и переносы %%%
%% http://tex.stackexchange.com/questions/241343/what-is-the-meaning-of-fussy-sloppy-emergencystretch-tolerance-hbadness
%% http://www.latex-community.org/forum/viewtopic.php?p=70342#p70342
\tolerance 1414
\hbadness 14145
\emergencystretch 1.5em % В случае проблем регулировать в первую очередь
\hfuzz 0.3pt
\vfuzz \hfuzz
%\raggedbottom
%\sloppy                 % Избавляемся от переполнений
\clubpenalty=10000      % Запрещаем разрыв страницы после первой строки абзаца
\widowpenalty=10000     % Запрещаем разрыв страницы после последней строки абзаца
\brokenpenalty=4991     % Ограничение на разрыв страницы, если строка заканчивается переносом
%%=====(end)[ĐỊNH DẠNG KHỔ GIẤY]==============================================>>>>>>>>>>>>>>>>>>>


%%=====(begin) [MODIFY package: memoir]========================================
%%% Оформление заголовков глав, разделов, подразделов %%%
%% Работа должна быть выполнена ... размером шрифта 12-14 пунктов (ГОСТ Р 7.0.11-2011, 5.3.8). То есть не должно быть надписей шрифтом более 14. Так и поставим.
%% Эти установки будут давать одинаковый результат независимо от выбора базовым шрифтом 12 пт или 14 пт
\newcommand{\basegostsectionfont}{\fontsize{14pt}{16pt}\selectfont\bfseries}
% Заголовки отделяют от текста сверху и снизу 3 интервалами (ГОСТ Р 7.0.11-2011, 5.3.5)
\newcommand*{\intafterskip}{24pt}	%maybe
\makechapterstyle{thesisgost}{%
	%\chapterstyle{default}
	\renewcommand*{\chapterheadstart}{}	% dịch chapter lên vị trí header (đưa lên cao nhất)
	%\setlength{\beforechapskip}{0pt} % length, tự giải thích, thường được đặt bằng \chapterheadstart, mặc định 50pt
	%\setlength{\midchapskip}{0pt} %length, khoảng cách giữa tên/số chương và tiêu đề, thường được đặt bằng cách sử dụng \afterchapternum, mặc định là 20p
	\setlength{\afterchapskip}{\intafterskip} %length, khoảng cách giữa tiêu đề chương và văn bản theo sau, thường được đặt bằng \afterchaptertitle, mặc định là 40pt
	% 3 cái này để đưa chapter [numbering]. title về cùng size 14, bold
	\renewcommand*{\chapnamefont}{\basegostsectionfont}	 %set font cho chữ chapter.
	\renewcommand*{\chapnumfont}{\basegostsectionfont}	 %set font cho number của chapter. e.g 1
	\renewcommand*{\chaptitlefont}{\basegostsectionfont} %set font cho title của chapter.
	\renewcommand*{\afterchapternum}{.\space}   % thêm dấu chấm có khoảng trắng sau số phần
	\renewcommand*{\printchapternum}{\centering\basegostsectionfont\thechapter} % căn giữa chapter
	\renewcommand*{\printchapternonum}{\centering}
}
\setsecnumformat{\csname the#1\endcsname.\space} % thêm . và space vào sau number của section. e.g 1.1. title
\settocdepth{subsection}    % chỉ mục đến subsection, e.g. 1.1.1
\setsecnumdepth{subsection} % до какого уровня нумеровать подразделы
% set fontsize và align (centering) cho section.
\setsecheadstyle{\basegostsectionfont\centering}
%\setsecindent{\otstuplen}
% set fontsize và align (centering) cho subsection.
\setsubsecheadstyle{\basegostsectionfont\centering}
%\setsubsecindent{0pt}
% set fontsize và align (centering) cho subsubsectisusu (ko dùng)
%\setsubsubsecheadstyle{\basegostsectionfont\centering}
%\setsubsubsecindent{0pt}
\sethangfrom{\noindent #1} %все заголовки подразделов центрируются с учетом номера, как block
\chapterstyle{thesisgost}

%%% Khoảng cách giữa các tiêu đề (trước và sau)
% Заголовки отделяют от текста сверху и снизу тремя интервалами (ГОСТ Р 7.0.11-2011, 5.3.5).
\setbeforesecskip{\intafterskip}
\setaftersecskip{\intafterskip}
\setbeforesubsecskip{\intafterskip}
\setaftersubsecskip{\intafterskip}
\setbeforesubsubsecskip{\intafterskip}
\setaftersubsubsecskip{\intafterskip}

%%% Колонтитулы %%%
% Порядковый номер страницы печатают на середине верхнего поля страницы (ГОСТ Р 7.0.11-2011, 5.3.8)
% ĐÁNH SỐ TRANG (PAGE) LÊN PHÍA TRÊN, CHÍNH GIỮA.
\makeevenhead{plain}{}{\rmfamily\thepage}{}
\makeoddhead{plain}{}{\rmfamily\thepage}{}
\makeevenfoot{plain}{}{}{}
\makeoddfoot{plain}{}{}{}
\pagestyle{plain}

%%%% PHẦN MỤC LỤC - CONTENTS
\renewcommand*{\cftappendixname}{\appendixname\space} % Слово Приложение в оглавлении
\renewcommand*{\cftchaptername}{\chaptername\space} % chữ Chapter sẽ được viết trước mỗi số phần trong mục lục
% thêm chấm (...) vào chapter ở phần mục lục (contents)
\renewcommand{\cftchapterdotsep}{\cftdotsep}                % отбивка точками до номера страницы начала главы/раздела
%% Переносить слова в заголовке не допускается (ГОСТ Р 7.0.11-2011, 5.3.5). Заголовки в оглавлении должны точно повторять заголовки в тексте (ГОСТ Р 7.0.11-2011, 5.2.3). Прямого указания на запрет переносов в оглавлении нет, но по той же логике невнесения искажений в смысл, лучше в оглавлении не переносить:
\setrmarg{2.55em plus1fil}                             %To have the (sectional) titles in the ToC, etc., typeset ragged right with no hyphenation
\renewcommand{\cftchapterpagefont}{\normalfont}        % нежирные номера страниц у глав в оглавлении
\renewcommand{\cftchapterleader}{\cftdotfill{\cftchapterdotsep}}% нежирные точки до номеров страниц у глав в оглавлении
%\renewcommand{\cftchapterfont}{}                       % нежирные названия глав в оглавлении
\renewcommand\cftchapteraftersnum{.\space}       % добавляет точку с пробелом после номера раздела в оглавлении
\renewcommand\cftsectionaftersnum{.\space}       % добавляет точку с пробелом после номера подраздела в оглавлении
\renewcommand\cftsubsectionaftersnum{.\space}    % добавляет точку с пробелом после номера подподраздела в оглавлении
\renewcommand\cftsubsubsectionaftersnum{.\space} % добавляет точку с пробелом после номера подподподраздела в оглавлении
%%%%% TAB indent
% Thụt lề đoạn đầu tiên sau tiêu đề phần - chỉ cần sử dụng 1 trong 2: \indentafterchapter hoặc \usepackage{indentfirst}
% default: section và subsection tự động thụt vào, chapter thì ko.
%Macro \indentafterchapter. From package: class-memoir.
\indentafterchapter     % Красная строка после заголовков типа chapter
%\usepackage{indentfirst} %Indent first paragraph after section header - 
\setlength{\parindent}{2.5em}                   % Абзацный отступ. Должен быть одинаковым по всему тексту и равен пяти знакам (ГОСТ Р 7.0.11-2011, 5.3.7).

%%=====(end) [MODIFY package: memoir]========================================
