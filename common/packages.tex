\usepackage{geometry}
\usepackage{comment}
\usepackage{totcount} %counter

%%% Математические пакеты %%%
\usepackage{amsthm,amsmath,amscd}   % Математические дополнения от AMS
\usepackage{amsfonts,amssymb}       % Математические дополнения от AMS
\usepackage{mathtools}              % Добавляет окружение multlined
\usepackage{xfrac}                  % Красивые дроби
\usepackage[
locale = DE,
list-separator       = {;\,},
list-final-separator = {;\,},
list-pair-separator  = {;\,},
list-units           = single,
range-units          = single,
range-phrase={\text{\ensuremath{-}}},
% quotient-mode        = fraction, % красивые дроби могут не соответствовать ГОСТ
fraction-function    = \sfrac,
separate-uncertainty,
]{siunitx}[=v2]                 % Размерности SI
\sisetup{inter-unit-product = \ensuremath{{}\cdot{}}}

%https://ctan.org/pkg/afterpage 
%– Execute command after the next page break
%- Để sử dụng \afterpage{\clearpage}  thay cho \clearpage??
% cơ bản giống nhau, ko khác gì mấy.
%%% Для добавления Стр. над номерами страниц в оглавлении
%%% http://tex.stackexchange.com/a/306950
%\usepackage{afterpage}

%%% enumitem – Control layout of itemize, enumerate, description
%https://mirror.macomnet.net/pub/CTAN/macros/latex/contrib/enumitem/enumitem.pdf
\usepackage{enumitem}
%To remove the vertical space altogether in all lists: \setlist{nosep}
% A first pattern aligns the label with the surrounding \parindent while the item body is indented depending on the label and a fixed labelsep: labelindent = \parindent, leftmargin = *
\setlist{ %% Каждый пункт, подпункт и перечисление записывают с абзацного отступа (ГОСТ 2.105-95, 4.1.8)
	nosep,
	labelindent=\parindent,
	leftmargin=*
}

%%% Таблицы %%%
\usepackage{longtable,ltcaption} % Длинные таблицы
\usepackage{multirow,makecell}   % Улучшенное форматирование таблиц
\usepackage{tabulary}      % таблицы с автоматически подбирающейся
\usepackage{threeparttable}      % автоматический подгон ширины подписи таблицы

% шириной столбцов (tabu обязательно до hyperref вызывать)
%%% Таблицы %%%
%\DeclareCaptionLabelSeparator{tabsep}{\tablabelsep} % нумерация таблиц
%\DeclareCaptionFormat{split}{\splitformatlabel#1\par\splitformattext#3}

%\usepackage[⟨options⟩]{caption}
\usepackage{caption}                % caption – Customising captions in floating environments https://ctan.org/pkg/subcaption
%https://github.com/kia999/babel-russian/blob/master/russianb.dtx#L958-L967
%2021-01-04 version 1.3m
%* The macro `\cyrdash` that prints Cyrillic dash has been changed.
% Now it is alias of `\textemdash` in all encodings.
%You can define your own caption label separators with\DeclareCaption-
%LabelSeparator \DeclareCaptionLabelSeparator{⟨name⟩}{⟨code⟩}
\DeclareCaptionLabelSeparator{tabsep}{\textemdash} % нумерация таблиц
% \captionsetup[⟨float type⟩]{⟨options⟩}
\captionsetup[table]{
	format=plain,                % формат подписи (plain|hang)
	font=normal,                      % нормальные размер, цвет, стиль шрифта
	skip=.0pt,                        % отбивка под подписью
	parskip=.0pt,                     % отбивка между параграфами подписи
	position=above,                   % положение подписи
	justification=centering, %justified,           % центровка
	indent=0cm,               	 		% смещение строк после первой
	labelsep=tabsep,%endash,                  % thay dấu : thành "--" (ГОСТ 2.105, 4.3.1)
	singlelinecheck=false, % не выравнивать по центру, если умещается в одну строку
}
% Sẽ tự động điều chỉnh bên Figure, ko cần cấu hình lại mỗi lần gọi: \figure
\captionsetup[figure]{
	format=plain,                     % формат подписи (plain|hang)
	font=normal,                      % нормальные размер, цвет, стиль шрифта
	skip=1.5pt,                       % khoảng cách giữa hình và caption. (default=0pt)
	parskip=.0pt,                     % отбивка между параграфами подписи
	position=below,                   % положение подписи
	singlelinecheck=true,             % выравнивание по центру, если умещается в одну строку
	justification=centerlast,         % центровка
	labelsep=tabsep,                  % Sử dụng "--" giống table.
}
%You can define your own caption fonts wit \DeclareCaptionFont{⟨name⟩}{⟨code⟩} .
% e.g. \DeclareCaptionFont{bf}{\bfseries}
%https://mirror.truenetwork.ru/CTAN/macros/latex/contrib/caption/subcaption.pdf
\usepackage{subcaption}             % Support for sub-captions
%\DeclareCaptionSubType[⟨numbering scheme⟩]{⟨type⟩
%%% Подписи подрисунков %%%
\DeclareCaptionSubType{figure} % thêm caption vào hình phụ.
% cấu hình 
\captionsetup[subfloat]{
	%labelfont=norm,                 % нормальный размер подписей подрисунков
	%textfont=norm,                  % нормальный размер подписей подрисунков
	labelsep=space,                 % разделитель
	labelformat=brace,              % одна скобка справа от номера
	justification=centering,        % центровка
	singlelinecheck=true,           % выравнивание по центру, если умещается в одну строку
	skip=.0pt,            			% отбивка над подписью
	parskip=.0pt,                   % отбивка между параграфами подписи
	position=below,                 % положение подписи
}
% The default numbering scheme is \alph, đánh theo English a), b) ...
\renewcommand{\thesubfigure}{\alph{subfigure}}
% Đánh số hình phụ theo tiếng Nga như: а), б), ...
%\renewcommand\thesubfigure{\asbuk{subfigure}} % нумерация подрисунков	


%%% Изображения %%%
\usepackage{graphicx}
\graphicspath{{images/}{Dissertation/images/}}         % Пути к изображениям


\usepackage{xcolor}
%using: \textcolor{⟨color ⟩}{⟨text⟩}; \color{⟨color}
% \definecolor[⟨type⟩]{⟨name⟩}{⟨model-list⟩}{⟨spec-list⟩}, e.g. \definecolor{red}{rgb}{1,0,0},
%\usepackage[dvipsnames, table]{xcolor} % Совместимо с tikz
\usepackage{tikz}                   % Продвинутый пакет векторной графики
%\usetikzlibrary{chains}             % Для примера tikz рисунка
%\usetikzlibrary{shapes.geometric}   % Для примера tikz рисунка
%\usetikzlibrary{shapes.symbols}     % Для примера tikz рисунка
%\usetikzlibrary{arrows}             % Для примера tikz рисунка

%SIÊU LIÊN KẾT
%https://latexcolor.com/
\definecolor{linkcolor}{rgb}{0.9,0,0} % hơi đỏ.
\definecolor{citecolor}{rgb}{0,0.6,0} % xanh lá nhạt dịu
\definecolor{urlcolor}{rgb}{0,0,1}	  % blue

\usepackage{hyperref}
\hypersetup{
	linktocpage=true,           % ссылки с номера страницы в оглавлении, списке таблиц и списке рисунков
	plainpages=false,           % Forces page anchors to be named by the Arabic form  of the page number, rather than the formatted form
	colorlinks=true,                 % default=false. Nếu =true, thì các ô vuông xung quanh sẽ mất, thay vào đó là màu sắc tương ứng cho các link, cite, url.
	linkcolor=linkcolor,%red,      % default: red (color of links ref, eqref, ...)
	citecolor=citecolor, %green,    % default: green, color of citation links
	urlcolor=blue, %magenta,%[rgb]{0,0,1}          % magenta Color for linked URLs (default)
	pdftitle={PhD Thesis},    % Заголовок
	pdfauthor={Dang Hien Ngo},  % Автор
	%pdfsubject={2.3.1 Control in system},      % Тема
	pdfcreator={hiennd},     % Создатель, Приложение
	%pdfproducer={Производитель},% Производитель, Производитель PDF
	%pdfkeywords={},    % Ключевые слова
	pdflang={en},
}
%\usepackage{bookmark} %\belowpdfbookmark{text1}{contents}

%https://ctan.altspu.ru/macros/latex/contrib/pdfpages/pdfpages.pdf
\usepackage{pdflscape}		% pdfpages package requires the following packages
\usepackage{pdfpages}       %This package simplifies the insertion of external multi-page PDF file.

\usepackage[linesnumbered,ruled,vlined]{algorithm2e}
%%% Coloring the comment as blue
\newcommand\mycommfont[1]{\footnotesize\ttfamily\textcolor{blue}{#1}}
\SetCommentSty{mycommfont}
\SetKwInput{KwInput}{Input}                % Set the Input
\SetKwInput{KwOutput}{Output}              % set the Output
%end----algorithm2e

\usepackage{listings} %Inserting code in this LaTeX document
\usepackage{matlab-prettifier} %https://www.overleaf.com/learn/latex/Questions/How_can_I_include_MATLAB_code_in_my_LaTeX_document%3F
%решаем проблему с кириллицей в комментариях (в pdflatex) https://tex.stackexchange.com/a/103712
\lstset{extendedchars=true,keepspaces=true,literate={Ö}{{\"O}}1
	{Ä}{{\"A}}1
	{Ü}{{\"U}}1
	{ß}{{\ss}}1
	{ü}{{\"u}}1
	{ä}{{\"a}}1
	{ö}{{\"o}}1
	{~}{{\textasciitilde}}1
	{а}{{\selectfont\char224}}1
	{б}{{\selectfont\char225}}1
	{в}{{\selectfont\char226}}1
	{г}{{\selectfont\char227}}1
	{д}{{\selectfont\char228}}1
	{е}{{\selectfont\char229}}1
	{ё}{{\"e}}1
	{ж}{{\selectfont\char230}}1
	{з}{{\selectfont\char231}}1
	{и}{{\selectfont\char232}}1
	{й}{{\selectfont\char233}}1
	{к}{{\selectfont\char234}}1
	{л}{{\selectfont\char235}}1
	{м}{{\selectfont\char236}}1
	{н}{{\selectfont\char237}}1
	{о}{{\selectfont\char238}}1
	{п}{{\selectfont\char239}}1
	{р}{{\selectfont\char240}}1
	{с}{{\selectfont\char241}}1
	{т}{{\selectfont\char242}}1
	{у}{{\selectfont\char243}}1
	{ф}{{\selectfont\char244}}1
	{х}{{\selectfont\char245}}1
	{ц}{{\selectfont\char246}}1
	{ч}{{\selectfont\char247}}1
	{ш}{{\selectfont\char248}}1
	{щ}{{\selectfont\char249}}1
	{ъ}{{\selectfont\char250}}1
	{ы}{{\selectfont\char251}}1
	{ь}{{\selectfont\char252}}1
	{э}{{\selectfont\char253}}1
	{ю}{{\selectfont\char254}}1
	{я}{{\selectfont\char255}}1
	{А}{{\selectfont\char192}}1
	{Б}{{\selectfont\char193}}1
	{В}{{\selectfont\char194}}1
	{Г}{{\selectfont\char195}}1
	{Д}{{\selectfont\char196}}1
	{Е}{{\selectfont\char197}}1
	{Ё}{{\"E}}1
	{Ж}{{\selectfont\char198}}1
	{З}{{\selectfont\char199}}1
	{И}{{\selectfont\char200}}1
	{Й}{{\selectfont\char201}}1
	{К}{{\selectfont\char202}}1
	{Л}{{\selectfont\char203}}1
	{М}{{\selectfont\char204}}1
	{Н}{{\selectfont\char205}}1
	{О}{{\selectfont\char206}}1
	{П}{{\selectfont\char207}}1
	{Р}{{\selectfont\char208}}1
	{С}{{\selectfont\char209}}1
	{Т}{{\selectfont\char210}}1
	{У}{{\selectfont\char211}}1
	{Ф}{{\selectfont\char212}}1
	{Х}{{\selectfont\char213}}1
	{Ц}{{\selectfont\char214}}1
	{Ч}{{\selectfont\char215}}1
	{Ш}{{\selectfont\char216}}1
	{Щ}{{\selectfont\char217}}1
	{Ъ}{{\selectfont\char218}}1
	{Ы}{{\selectfont\char219}}1
	{Ь}{{\selectfont\char220}}1
	{Э}{{\selectfont\char221}}1
	{Ю}{{\selectfont\char222}}1
	{Я}{{\selectfont\char223}}1
	{і}{{\selectfont\char105}}1
	{ї}{{\selectfont\char168}}1
	{є}{{\selectfont\char185}}1
	{ґ}{{\selectfont\char160}}1
	{І}{{\selectfont\char73}}1
	{Ї}{{\selectfont\char136}}1
	{Є}{{\selectfont\char153}}1
	{Ґ}{{\selectfont\char128}}1
}
